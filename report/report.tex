\documentclass[sigconf,nonacm]{acmart}

\bibliographystyle{alpha}
\title{Geodesic Paths and Distances\\{\normalsize Report on \textit{A Survey of Algorithms for Geodesic Paths and Distances}}}

\author{Matthieu Pierre Boyer}
\authornote{Both authors contributed equally.}
\affiliation{%
	\institution{École Normale Supérieure}
	\city{Paris}
	\country{France}}
\email{matthieu.boyer@ens.fr}

\author{Antoine Groudiev}
\authornotemark[1]
\affiliation{%
	\institution{École Normale Supérieure}
	\city{Paris}
	\country{France}}
\email{antoine.groudiev@ens.fr}

\def\abs#1{\left|#1\right|}
\def\R{\mathbb{R}}

\def\V{\mathbb{V}}
\def\nv{n_{\V}}
\def\E{E}
\def\ne{n_{\E}}
\def\F{\mathcal{F}}
\def\nf{n_{\F}}
\def\d{\mathrm{d}}

\begin{abstract}
	In this report, we look at different methods to compute shortest-paths
	on meshed $2$-manifolds embedded in $\R^{3}$, based on \cite{craneSurvey}.
	We will most notably compare different types of methods, either coming from
	the resolution of PDEs on the manifold and unfolding the embedding to $\R^{2}$.
	We will also provide benchmarks on the different methods, implemented in Rust.
	% for a blazingly fast™ result.
\end{abstract}

\keywords{Geodesic, Paths, Distances}

\begin{teaserfigure}
	% on peut remplacer l'image par quelque chose de plus horizontal du même genre, ou mettre plusieurs images côte à côte
	\centering
	\includegraphics[width=.4\textwidth]{images/rabbit-paths.png}
	\caption{Geodesic paths on a meshed $2$-manifold computed with our implementation of the Improved Chen-Han algorithm. Faces are colored from red to green according to the distance of their barycenter to the source point. The geodesic path to each vertex is drawn in blue.}
	\Description{}
	\label{fig:teaser}
\end{teaserfigure}

\begin{document}
\maketitle

\section*{Introduction}

\section{Mathematical reminders}
A $2$-manifold (without boundary) is a topological space in which all points have nieghbourhoods homeomorphic
to disks (without boundary) in $\R^{2}$.
This means that zooming enough on every point looks like the plane.

In our context, we will be given a $2$-manifold already meshed\footnote{\cite{craneSurvey} gives a few
	methods to create such a meshing.}, that is, a finite set $\V \subseteq \R^{3}$ of vertices
(of cardinal $\nv$) and a finite set $\F \subseteq \left\llbracket 1, \nf\right\rrbracket^{3}$
of faces, given by the indices of the associated vertices.
The edges $\E$ of the manifold are given by any subsets of size two of $f \in \F$.
Because we can't have continuous functions, functions on the manifold will be represented as functions
on $\V$, on $\E$ or on $\F$.
As such, we will take any function and interpolate it linearly on each face, giving us a piecewise-linear
function.

A path on a piecewise-linear manifold can then be understood as interpolating on pieces of the manifold, or
directly computing the curves on the mesh.
The quality of the approximation by the mesh of the $2$-manifold will never be taken into account in the
quality results.

\section{PDE-based methods}
In this section we will study methods inspired by partial differential equations that arise from models
of physical phenomena.
Indeed, many physical phenomena propagate along the surfaces over time, and dissipate over space, thus
allowing us to retrieve geodesics from solutions to the equations.

\subsection{General Theory}
Consider the parabolic heat equation
\begin{equation*}
	\frac{\d}{\d t}u_{t} = \Delta u_{t}.
\end{equation*}
Here $u_{t}$ is the temperature profile at time $t$ and $\Delta$ is the laplacian operator (or divergence of the gradient operator).
However, on a piecewise-linear manifold, because functions on vertices are interpolated linearly to become
functions on the manifold, the gradient is piecewise-constant and can be computed explicitly from the
values at each vertex.
Consider a face $f \in \F$ with vertices $p_{i}, p_{j}, p_{k} \in \R^{3}$.
Let $e_{1} = p_{j} - p_{i}$ and $e_{2} = p_{k} - p_{i}$.
The face normal is
\begin{equation*}
	n_{f} = \frac{e_{1} \times e_{2}}{\abs{e_{1} \times e_{2}}}
\end{equation*}
where $\times$ is the cross product in $\R^{3}$
and thus the gradient, being perpendicular to level curves is
\begin{equation*}
	\nabla u_{\mid_{f}} = \frac{1}{2A_{f}}\sum_{l \in f}u_{l}(n_{f}\times e_{l})
\end{equation*}
where $A_{f} = \frac{1}{2}\abs{e_{1} \times e_{2}}$ is the area of face $f$ and $e_{l}$ is the \textbf{opposite} to vertex $l$.
Note that we can represent the gradient as a matrix $G \in \R^{\nv \times 3\nf}$, although this
representation is really inefficient for practical computation.
The definition of the divergence then comes from the Gauss-Ostrogradski theorem by intregration by parts
\begin{equation*}
	\left(\nabla \cdot X\right)_{i} = \frac{1}{A_{i}}\sum_{f \ni i}\sum_{e \in f}\frac{1}{2}\cot{\left(\alpha_{e}^{f}\right)}\langle X_{f}, e\rangle
\end{equation*}
where $A_{i}$ is the Voronoi area associated with vertex $i$ and $\alpha_{e}^{f}$ is the angle at the vertex
opposite to edge $e$ in $f$.

Finally, we can define the Laplace-Beltrami operator (the piecewise-linear version of the continuous laplacian) as $\Delta = (\nabla \cdot) \circ \nabla : \V \to \R$ or simply
\begin{equation*}
	(\Delta u)_{i} = \frac{1}{2A_{i}}\sum_{e = (i, j)}\left(\cot \alpha_{i, j} + \cot \beta_{i, j}\right)\left(u_{i} - u_{j}\right)
\end{equation*}
where $\alpha_{i, j}, \beta_{i, j}$ are the two angles opposite edge $(i, j)$.
One could then see the Laplace-Beltrami operator as a matrix $L \in \R^{\nv \times \nv}$.

\smallskip

After the spatial discretization of the laplace operator we just described, operating a time discretization in a single
backward Euler step for some fixed time $t$ will give us approximate solutions to the equation.
If we want to find the distance maps from some set $X \subseteq \V$, we simply solve the linear equation
associated to the continuous equation we need to solve.

\subsection{Implementations}
In our implementation, we compared multiple methods based on physical phenomena allowing to trace geodesics,
or at least geodesic-like curves.
Indeed, not all of those compute the true geodesic distance, but some sense of distance that can be drawn
and integrated to find shortest paths.

\paragraph{Heat Method}
This method is based on the heat equation $\nabla u_{t} = \frac{\d}{\d t}u_{0}$, which models the
evolution of temperature profiles $u_{t}$ in time in a given material, which here will be our surface,
from the initial profil $u_{0}$.
It can be derived
We discretize it as:
\begin{equation*}
	(\mathrm{id} - t\Delta)u_{t} = u_{0}
\end{equation*}
Retrieving the true geodesic distance can the be done by first normalizing the gradient
$X = -\nabla u_{t}/\abs{\nabla u_{t}}$ of the solution of the above equation that points along geodesics,
then solving the Poisson equation $\Delta \phi = \nabla \cdot X$ to retrieve the true distance function.
This fact comes from the Varadhan formula $\phi(x, y) = \lim_{t \to 0}\sqrt{-4t \log k_{t, x}(y)}$.

\cite{Crane:2017:HMD} suggest that the proper value of $t$ to use for computations here is around $h^{2}$
with $h$ the mean spacing between adjacent nodes, as $h\Delta$ is invariant with respect to scale.

\paragraph{Poisson Equations}
The equations in this paragraph all allow to draw geodesics-like curves, although they do not give the
actual metric on the manifold.
They are all based on the Poisson equation
\begin{equation*}
	\Delta u = u_{0},
\end{equation*}
which can be derived, for example, from the Maxwell equations to compute the electrostatic potential
along a charge distribution, or from the momentum equation to compute the pressure in a incompressible
fluid given its velocity.
% TODO: implement and provide informations


\paragraph{Wavefront Propagation}
The hyperbolic wavefront propagation equation
\begin{equation*}
	\frac{\d^{2}}{\d t^{2}}u_{t} = c^{2}\Delta u_{t}, \text{or}, \square u = 0,
\end{equation*}
where $\square$ is the d'Alembert operator, models the propagation of a wave in a material,
which will again be our surface here.
It arises for example the response of the surface to some elastic deformation $u$ considering the stress tensor
$T = E\nabla u$ with $E$ the homogenous modulus of elasticity, and considering the inertial force
$\rho \frac{\partial^{2}u}{\partial t^{2}}$ caused by the local acceleration.


\section{Improved Chen-Han}


\appendix
\bibliography{report}
\end{document}
