\documentclass[sigconf,nonacm]{acmart}

% TODO: remove this from final version (ptdr @antoine pourquoi tu l'as pas encore enlevé ça ?)
\setcopyright{acmlicensed}
\copyrightyear{2025}
\acmYear{2025}
\acmDOI{XXXXXXX.XXXXXXX}
\acmConference[]{}{}{}
\acmISBN{}
\bibliographystyle{alpha}

\begin{document}

\title{Geodesic Paths and Distances\\[0\baselineskip]{\small Report on \textit{A Survey of Algorithms for Geodesic Paths and Distances}}}

\author{Matthieu Boyer}
\authornote{Both authors contributed equally.}
\affiliation{%
	\institution{École Normale Supérieure}
	\city{Paris}
	\country{France}}
\email{matthieu.boyer@ens.fr}

\author{Antoine Groudiev}
\authornotemark[1]
\affiliation{%
	\institution{École Normale Supérieure}
	\city{Paris}
	\country{France}}
\email{antoine.groudiev@ens.fr}

\def\R{\mathbb{R}}
\def\V{\mathbb{V}}
\def\nv{n_{\V}}
\def\E{E}
\def\ne{n_{\E}}
\def\F{\mathcal{F}}
\def\nf{n_{\F}}

\begin{abstract}
	In this report, we look at different methods to compute shortest-paths
	on meshed $2$-manifolds embedded in $\R^{3}$, based on \cite{craneSurvey}.
	We will most notably compare different types of methods, either coming from
	the resolution of PDEs on the manifold and unfolding the embedding to $\R^{2}$.
	We will also provide benchmarks on the different methods, implemented in Rust.
	% for a blazingly fast™ result.za
\end{abstract}

\keywords{Geodesic, Paths, Distances}

\begin{teaserfigure}
	% ce serait bien d'avoir une image ici
	%   \includegraphics[width=\textwidth]{sampleteaser}
	\caption{}
	\Description{}
	\label{fig:teaser}
\end{teaserfigure}


\maketitle

\section*{Introduction}

\section{Mathematical reminders}
A $2$-manifold (without boundary) is a topological space in which all points have nieghbourhoods homeomorphic
to disks (without boundary) in $\R^{2}$.
This means that zooming enough on every point looks like the plane.

In our context, we will be given a $2$-manifold already meshed\footnote{\cite{craneSurvey} gives a few
	methods to create such a meshing.}, that is, a finite set $\V \subseteq \R^{3}$ of vertices
(of cardinal $\nv$) and a finite set $\F \subseteq \left\llbracket 1, \ldots, \nf\right\rrbracket^{3}$
of faces, given by the indices of the associated vertices.
The edges $\E$ of the manifold are given by any subsets of size two of $f \in \F$.
Because we can't have continuous functions, functions on the manifold will be represented as functions
on $\V$, on $\E$ or on $\F$.
As such, we will take any function and interpolate it linearly on each face, giving us a piecewise-linear
function.

A path on a piecewise-linear manifold can then be understood as interpolating on pieces of the manifold, or
directly computing the curves on the mesh.
The quality of the approximation by the mesh of the $2$-manifold will never be taken into account in the
quality results.

\section{PDE-based methods}
\subsection{Theory}

\subsection{Implementations}

\section{Improved Chan-Han}


\appendix
\bibliography{report}
\end{document}
